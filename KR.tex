\documentclass[russian, utf8, nocolumnxxxi, nocolumnxxxii]{eskdtext}
\usepackage[utf8]{inputenc}
\usepackage[T1,T2A]{fontenc}
\usepackage[english, russian]{babel}
\usepackage[argument]{graphicx}
\usepackage{graphicx}
\graphicspath{{images/}}
\usepackage{wrapfig}
\usepackage{tikz}
\usepackage{siunitx}
\usepackage[american,cuteinductors,smartlabels]{circuitikz}

\title{KR}
\author{danilaturygin}
\date{December 2018}

\begin{document}
1. Даны функции f(x) = $\sqrt{3}$sin(x) + cos(x); \\
g(x) = cos(2x + $\pi$/3) - 1 \\
а)Решить уравнение f(x)=g(x). \\
б)Исследовать функцию h(x) = f(x) - g(x) на промежутке [0; $\frac{5}{6}$$\pi$] \\
Строим график функции:\\
function y=h(x) \\
y=sqrt(3)*sin(x)+cos(x)-cos(2*x+pi/3)+1 \\
endfunction \\
plot(0:0.01:2*pi,h) \\
\includegraphics[width=\textwidth]{1.1} \\
Рисунок 1 \\
Находим корни уравнения: \\
deff($'y=h(x)'$,$'y=sqrt(3)*sin(x)+cos(x)-cos(2*x+$pi$/3)+1'$) \\
x0=[3,4.2,4.3,5.5]; \\
$[x,v]$=fsolve(x0,h) \\
 $v$  = \\
-2.220D-16 \quad 0. \quad 0. \quad 7.772D-16 \\
 $x$  = \\
   2.6179939 \quad 4.1887902 \quad 4.1887902 \quad 5.759586 \\
3*x/$pi$ \\
 ans  = \\
   2.5 \quad 4. \quad 4. \quad 5.5\\
Корни уравнения:\\
$$x_1 = \frac{5}{6}\pi + 2n\pi, n \in Z$$
$$x_2 = \frac{8}{6}\pi + 2n\pi, n \in Z$$
$$x_3 = \frac{11}{6}\pi + 2n\pi, n \in Z$$


\end{document}
